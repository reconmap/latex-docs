\documentclass{beamer}
\usetheme{Boadilla}
\usepackage[utf8]{inputenc}

\usepackage{hyperref}
\hypersetup{
    colorlinks=true,
    linkcolor=blue,
    filecolor=magenta,      
    urlcolor=cyan,
}

\usepackage{dirtytalk}

\title{Democratising Security practices}
\subtitle{with \textbf{Reconmap} pentest automation and reporting}
\author{Santiago Lizardo}
\date{\today}

\begin{document}

\begin{frame}
\titlepage    
\end{frame}

\begin{frame}{Outline}
    \tableofcontents
\end{frame}

\begin{frame}{\section{What is a pentest?}}
    \say{Penetration testing, also known as pentesting, is the practice of testing a computer system, network or web application to find security vulnerabilities that an attacker could exploit. Penetration testing can be automated with software applications or performed manually.}
    \href{https://pentestreports.com/what-is-a-pentest.html}{https://pentestreports.com/what-is-a-pentest.html}
\end{frame}

\begin{frame}{\section{Who is a pentester?}}
    \say{A pentester, or penetration tester, is an individual who identifies security flaws within a network or system. They are often external consultants, authorized by a company to perform security audits on their IT ecosystem, and identify any potential cybersecurity risks.}
    \href{https://pentestreports.com/who-is-a-pentester.html}{https://pentestreports.com/who-is-a-pentester.html}
\end{frame}

\section{What is Reconmap?}

\begin{frame}{What is Reconmap?}
    \includegraphics[width=0.5\textwidth]{reconmap-logo.png}
    \begin{itemize}
        \item Pentest automation and reporting tool
        \item Open-source (\href{https://github.com/reconmap}{code})
        \item Makes pentesting accessible to all IT pros (developers, devops, sysadmins, ...)
    \end{itemize}
\end{frame}

\section{How does Reconmap work?}

\begin{frame}{How does Reconmap work?}
    \begin{enumerate}
        \item Web application is used to create engagement details
        \item \href{https://github.com/Reconmap/cli}{CLI tool} runs commands and pushes results to the API
        \item A pentest report is automatically generated
    \end{enumerate}
\end{frame}

\section{Reconmap feature set}

\begin{frame}{Reconmap feature set}
    \begin{itemize}
        \item Client, project, tasks management all in one.
        \item Reusable project templates and vulnerability management.
        \item Can scale to teams and projects of any size.
        \item Includes user roles, search, data export/import, ...
    \end{itemize}
\end{frame}

\section{How to get started?}

\begin{frame}{How to get started?}
    \begin{columns}[T]
        \begin{column}{.45\textwidth}
            \begin{block}{Manual setup}
            Follow \href{https://reconmap.org/admin-manual/}{setup instructions}\\
			\bigskip
            Requires significant time to install and maintain\\
            Community support (chat)
            \end{block}
        \end{column}
        \begin{column}{.45\textwidth}
            \begin{block}{SaaS}
            \href{https://reconmap.com}{Affordable hosting}\\
            \bigskip
            Ready in minutes\\
            Technical support (phone, email, chat)
            \end{block}
        \end{column}
    \end{columns}
    
\end{frame}

\section{Staying in touch}

\begin{frame}{Staying in touch}
    \begin{itemize}
        \item \href{https://github.com/reconmap}{Github} community
        \item \href{https://twitter.com/reconmap}{Twitter} updates
        \item \href{https://facebook.com/reconmap}{Facebook}
        \item \href{https://gitter.im/reconmap/community}{Gitter} chat
    \end{itemize}
\end{frame}

\end{document}

